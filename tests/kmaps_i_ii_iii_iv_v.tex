
\documentclass[a4paper,12pt,fleqn]{article}
%% Set page layout
\usepackage[a4paper,bindingoffset=0.2in,left=.7in,right=0.7in,top=1in,bottom=1in,footskip=.40in]{geometry}
%\usepackage{calc}
%\usepackage{mathptmx}
\usepackage[scaled=0.92]{helvet}
\usepackage{sansmath}
\usepackage{xcolor}
\usepackage{printlen}
\usepackage{tikz}

% Use American Style K-maps package...
\usepackage{askmaps}
\askmapunitlength=8.8mm

\definecolor{red}{rgb}{1,0,0}
\definecolor{blue}{rgb}{0,0,1}
\definecolor{darkgreen}{rgb}{0,0.4,0}
\definecolor{orange}{rgb}{1,0.5,0}
%
\thispagestyle{empty}
\setlength\parindent{0pt}

\input{kvmacros.tex}

\begin{document}

%Test: {\slshape 0123456789} {\itshape 0123456789}
\newpage
\section*{American Style Karnaugh Maps}

The Karnaugh maps for one variable
\medskip

\askmapi{S}{a}{f}{10}{}
\askmapi{S}{a}{fi}{01}{}
\bigskip

The Karnaugh maps for two variables
\medskip

\askmapii{S}{ab}{f}{1010}{}
\askmapii{S}{ab}{fi}{1010}{}
\bigskip

The Karnaugh maps for three variables
\medskip

\askmapiii{S}{abc}{f}{00111010}{}
\askmapiii{S}{ab{$x^n$}}{fi}{00111010}{}
\bigskip

The Karnaugh maps for three variables (alternate style)
\medskip

\askmapiiialt{S}{abc}{f}{00111010}{}
\askmapiiialt{S}{abd}{fi}{00111010}{}
\bigskip

The Karnaugh maps for four variables
\medskip

\askmapiv{S}{abcd}{f}{1100100100011101}{}
\askmapiv{S}{abcd}{fi}{1100100100011101}{}
\bigskip

\newpage
The Karnaugh maps for five variables
\medskip

\askmapv{S}{abcde}{fi}{01010101101010100101010110101010}{}
\bigskip

Or you can split the 5-map into two 4-maps with additional information
at the bottom...
\medskip

\askmapiv{S}{bcde}{fi}{0101010110101010}{%
\put(2,-0.8){\makebox(0,0){a = 0}}}
\askmapiv{S}{bcde}{fi}{0101010110101010}{%
\put(2,-0.8){\makebox(0,0){a = 1}}}
\bigskip




\newpage
You can also do math things by using the known \$ signs...
\medskip

\askmapii{$S_{0}$}{{$a_{1}$}{$a_{0}$}}{f}{1010}{}
\bigskip

You can do math thing in roman font...
\medskip

\askmapiii{$\mathrm{M^{n+1}_{0}}$}{{$\mathrm{M^{n}_{2}}$}{$\mathrm{M^{n}_{1}}$}{$\mathrm{M^{n}_{0}}$}}{f}{11100111}{}%
\bigskip

You can do things with don't cares...
\medskip

\askmapii{S}{ab}{f}{011-}{}
\askmapiii{S}{abc}{f}{001--10-}{}
\askmapiv{S}{abcd}{fi}{110--001-0001-110-}{}
\bigskip

You can use colors and empty function values and variables as values too...
\medskip

\askmapiii{S}{abc}{f}{{\color{blue}{0}}{\color{blue}{0}}{\color{red}{1}}{ }{ }{\color{red}{1}}{\color{blue}{0}}{ }}{}
\askmapii{S}{ab}{f}{{$i_{0}$}{$i_{1}$}{$i_{2}$}{$i_{3}$}}{}
\bigskip

\newpage
You can set the font to something else ... and use sans math font ...
\medskip

{\fontfamily{phv}\selectfont%
\askmapiv{S}{abcd}{f}{0110111011110011}{}%
}
{\fontfamily{phv}\selectfont\sansmath
\askmapiv{$Q^{n+1}_{0}$}{ {$p^{n}_{1}$} {$p^{n}_{0}$} {$q^{n}_{1}$} {$q^{n}_{0}$} }{f}{0110111011110011}{}%
}
\bigskip

You can use the last parameter to create to your own picture commands...
\medskip

{\fontfamily{phv}\selectfont\sansmath
\askmapiii{F}{xyz}{f}{11100111}{%
\color{red}\put(0.1,0.1){\dashbox{0.1}(0.8,1.8){}}%
\color{blue}\put(1.1,1.1){\dashbox{0.1}(1.8,0.8){}}%
\color{darkgreen}\put(2.1,0.1){\dashbox{0.1}(1.8,0.8){}}%
\color{orange}\put(0.15,1.15){\dashbox{0.1}(1.7,0.7){}}%
}}%
\bigskip

blabla
\medskip

\askmapii{F}{xy}{f}{0111}{%
\setlength\fboxsep{0pt}\linethickness{0pt}
\put(1.1,0.1){\pgfsetfillopacity{0.2}\colorbox{red}{\framebox(0.8,1.8){}}\pgfsetfillopacity{1}}%
\put(0.1,0.1){\pgfsetfillopacity{0.2}\colorbox{blue}{\framebox(1.8,0.8){}}\pgfsetfillopacity{1}}%
}

\askmapii{F}{xy}{f}{0111}{%
\begin{tikzpicture}[x=\askmapunitlength,y=\askmapunitlength]
\draw[red] (0,0) rectangle (2,2);
\end{tikzpicture}
}


\newpage
\section*{Bounding Boxes around the Maps}

{\setlength{\fboxsep}{0pt}%
\fbox{%
\askmapi{S}{a}{i}{01}{}%
}%

{\setlength{\fboxsep}{0pt}%
\fbox{%
\askmapii{S}{ab}{i}{0111}{}%
}%

{\setlength{\fboxsep}{0pt}%
\fbox{%
\askmapiii{S}{abc}{i}{01111111}{}%
}%

{\setlength{\fboxsep}{0pt}%
\fbox{%
\askmapiiialt{S}{abc}{i}{01111111}{}%
}%

{\setlength{\fboxsep}{0pt}%
\fbox{%
\askmapiv{S}{ab{x$^n$}{y$^n$}}{i}{0111111111111111}{}%
}%

{\setlength{\fboxsep}{0pt}%
\fbox{%
\askmapv{S}{abcll}{i}{01111111111111111111111111111111}{}%
}%

{\setlength{\fboxsep}{0pt}%
\fbox{%
\askmapv{$\mathrm{S_{a,b,c,d}}$}{abc{x$_0$}{x$_1^n$}}{i}{abcdefghijklmnopqrstuvwxyz0123456789}{}%
}%
Text next to Karnaugh map
\bigskip

%%% replacement for Karnaugh macros
%{\setlength{\fboxsep}{0pt}%
%\fbox{%
%\askmap{2}{S}{ab}{abcd}{}
%}%
%\bigskip

Next shows the use of \texttt{fbox} commands to show frameboxes.

{\setlength{\fboxsep}{0pt}%
{\fbox{%
\begin{minipage}[t]{0.7\linewidth}
\vspace{0pt}%
{\setlength{\fboxsep}{0pt}%
\fbox{%
%\askmapv{$\mathrm{F(a,b,c,d)}$}{abc{x$_0$}{x$_1^n$}}{i}{01100010100101010100101000111111}{}%
\askmapv{$\mathrm{M_1^{n+1}}$}{abc{x$_0$}{x$_1^n$}}{i}{01100010100101010100101000111111}{}%
}}%
\end{minipage}}\hfill
\fbox{%
\begin{minipage}[t]{0.3\linewidth}
This is text.\vspace{0pt}
\end{minipage}}}}

\bigskip\bigskip
Test to see in kvindex macro is working...
\kvindex

\askmap{1}{S}{ab}{abcd}{}

\bigskip
Test to see in kvnoindex macro is working...
\kvnoindex

\askmap{1}{S}{ab}{abcd}{}

\bigskip
The unit length of the Karnaugh map squares is
\printlength{\askmapunitlength}

\end{document}